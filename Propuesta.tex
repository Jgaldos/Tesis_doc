%\chapter{Nombre del Capítulo}
\chapter{Propuesta-version vieja}



\section{Parámetros a evaluar}
Para lograr nuestra meta dividimos la evaluación de \ac{EBGM} en los siguiente puntos


\section{Esquema de la propuesta}
A continuación mostramos un esquema que representa el resultado de la propuesta:

\begin{figure}
	\includegraphics[width=\textwidth]{Propuesta}
    \caption{Propuesta de modificación de \ac{EBGM}}
    \label{Proceso}
\end{figure}
La propuesta es agregar técnicas y modificaciones al curso normal de \ac{EBGM}. Para ello usamos la implementación ofrecida por Bolme \cite{bolme2003elastic} la cual puede ser entendida en la figura \ref{Proceso}:

Los puntos que son modificados son los siguientes: 
\begin{itemize}
\item Nueva selección de imágenes para los modelos de grafo con mayor variedad de rostros.
\item Modificación de los tamaños del conjunto de los Gabor Wavelet.
\item Agregar procesos de mejora de iluminación en la fase de pre-procesamiento del algoritmo.
\item Modificar la función de distancia para darle pesos a los diferentes puntos fiduciales.
\end{itemize}


%Una sección puede contener n sub secciones.\cite{Galante01}

%\subsubsection{Sub sub sección}

%Una sub sección puede contener n sub secciones.
%\section{Consideraciones Finales}

%Cada capítulo excepto el primero debe contener al finalizarlo una sección de consideraciones que enlacen
%el presente capítulo con el siguiente.
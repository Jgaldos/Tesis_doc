%\chapter{Conclusiones y Trabajos Futuros}\label{chap:conclusiones}
\chapter{Conclusiones }\label{chap:conclusiones}
%Las conclusiones de la tesis son una parte muy importante y tiene las siguientes partes.
%En primer lugar debes escribir las conclusiones generales de tu trabajo. evita escribirlas en forma de viñetas. Simplemente utiliza texto continuo.

Podemos ver que en la tabla \ref{Tcomparacion} y en la figura \ref{Fcomparacion} que \ac{EBGM} que es un algoritmo muy robusto, que mantienen porcentajes de aciertos muy buenos independiente de la base de datos en la que se realiza la prueba y que en ciertas situaciones es mejor que \ac{PCA}, y los otros métodos holísticos.

Hemos logrado mejoras con las modificaciones propuestas, podemos ver las modificaciones en las Gabor Wavelet mejoran el rendimiento del reconocimiento y en el peor de los casos no lo afecta el rendimiento original.El método de pre-procesamiento propuesto por Manjula\cite{manjulaimage} mejora el reconocimiento, algunas configuraciones de pesos afecta el reconocimiento pero siempre hay una configuración que mejora el reconocimiento.

Estos resultados nos acerca a un método de reconocimiento que es robusto para cualquier situación y que puede ser apto para el reconocimiento de rostros en vídeo vigilancia.

Debemos reemplazar la forma en que \ac{EBGM} con un método de detección de puntos mas fiable que los modelos actuales.

Es posible que en una futura investigación se pueda aprovechar el hecho que en ocasiones más de una cámara graba a un objeto desde diferentes ángulos lo cual brindaría mayor información sobre el rostro a reconocer.
La baja calidad de las imágenes es uno de los mayores retos ya que en muchos casos la cámara de seguridad no tiene una gran resolución para distancias largas y cuando una persona está lo bastante cerca el ángulo de visión aumenta de tal manera que se oculta casi la totalidad del rostro.
El hardware de la cámaras de vigilancia. muchas de las cámaras son fabricadas con sensores que no ofrecen las mejores condiciones para la cancelación de ruido. 
La naturaleza del ambiente no controlado, la adquisición de las imágenes son dependientes a muchos factores, tales como los individuos , la iluminación y muchos otros los cuales no controlamos

Finalmente hay que fijar un limite para la privacidad, en el caso de estas aplicación en la cual queremos reconocer un individuo en ambientes, los cuales puede ser lugares públicos existe la posibilidad de infringir la privacidad.

%La segunda  parte de este capítulo corresponde a las limitaciones que tiene la propuesta. Esta sección es muy importante para que los siguientes estudiantes que hagan algo en esta línea no cometan los
%mismos errores y tu tesis sea un buen peldaño para avanzar más rápido.

%\section{Recomendaciones}
%En esta sección el tesista debe reflejar que la tesis ha permitido adquirir nuevos conocimientos que podrían servir para guiar otros trabajos en el futuro.

%\section{Trabajos futuros}
%En base a los puntos anteriores es recomendable que tu tesis también sugiera trabajos futuros. Esta sección es esencialmente útil para otras ideas de tesis.

%Todo este capítulo no debe ser más de 4 páginas.
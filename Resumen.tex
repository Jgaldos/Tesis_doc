\begin{resumen}
%Aquí deberás colocar entre 100 y 150 palabras como
%máximo, el problema que intentas resolver, la
%justificación y los aportes o soluciones que planteas.

El reconocimiento de rostros es un área con una gran cantidad de aplicaciones y técnicas. 
Muchas de esas técnicas ofrecen buenos resultados cuando se aplican a situaciones donde el ambiente en el cual se desea realizar el reconocimiento es controlado, esto se entiende como el control de los factores que influyen en el proceso de reconocimiento, tales como iluminación, pose del rostro, expresión facial, etc.

Pero para el caso de ambientes no controlados, como lo es la vídeo vigilancia, el reconocimiento de rostros aún presenta dificultades: variación en la iluminación, falta de colaboración de las personas a reconocer, entre varios otros.
Debido a la importancia que tiene en seguridad y a la cantidad de infraestructura existente, es necesario aplicar el reconocimiento de rostros a video vigilancia.

%Para afrontar este problema este trabajo de tesis propone el un ``\textit{pipeline}'' de reconocimiento de rostros donde se usa una modificación de \ac{EBGM} usando \ac{CLNF} como reemplazo a la función de detección de puntos del algoritmo original, para finalmente ser probado en vídeos de cámaras de seguridad. 

Para afrontar los problemas mencionados, se propone un \textit{pipeline} de reconocimiento de rostros usando \ac{EBGM} con \ac{CLNF} como reemplazo a la función de detección de puntos del algoritmo original, para finalmente ser aplicado en vídeo. 
En este trabajo de tesis se realizó un análisis paramétrico de \ac{EBGM} para encontrar el factor mas influyente en su rendimiento, junto con su comparación con otros métodos de reconocimiento de rostros. También se determinó que elementos forman parte del \textit{pipeline} presentado como resultado final.

Finalmente la propuesta fue probada en una base de datos creada a partir de tomas de una cámara de seguridad, que consta de 24 sujetos con 8 imágenes cada uno. Los resultados finales muestra una mejora de 10\% y 6\% en imágenes tomadas en la mañana y en el medio día respectivamente

%Mientras varios métodos tratan el rostros como un todo, sin reparar las características que lo componen. Presentamos la mejora método que aborda las características no lineas a partir de puntos "fiduciales" que también pueden ser tratados como características biométricas. Este método basado en características es conocido como \ac{EBGM}.


%Para cumplir este objetivo se presenta una implementación del método de \acf{EBGM} masivamente paralelo.
\end{resumen}
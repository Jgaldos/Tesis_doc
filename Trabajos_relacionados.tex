\chapter{Trabajos relacionados}
%En este capítulo está destinado a explicar el estado del arte de las investigaciones relacionadas a la propuesta de tesis.
%Este capítulo debe contener un buen numero de referencias y no debe exceder 3 páginas de texto. Para esto se puede tomar como ejemplo la forma de hacer una revision bibliográfica en un
%articulo científico, por ejemplo, en la revista ACM Transactions on Graphics. Ese es el estilo que se debe adoptar.

%Al final del capítulo se debe explicar en donde encaja nuestra propuesta, respecto a las técnicas expuestas anteriormente
%En este capitulo describiremos brevemente los trabajos que poseen relevancia para poder entender la propuesta presentada en la ultima sección de este capítulo.
Como se menciona en los capítulos anteriores existe un área donde los métodos de reconocimiento de rostros y la biométria se intersecan \cite{}. Según los "survey" de la literatura sobre reconocimiento de rostros \cite{zhao2003face} y sobre biométria\cite{delac2004survey} esa área son lo métodos de reconocimiento basados en características. Mucho trabajos mencionan \ac{EBGM}\cite{wiskott1997face} como un método representativo de del reconocimiento de rostros \cite{}, también existen trabajos que comparan \ac{EBGM} con la manera en que los seres humanos percibimos e identificamos rostros \cite{bruce1998human} y como las Gabor wavelets que se usan en este método procesan la información de forma similar que ciertas partes del cerebro \cite{}

En el trabajo de Tseng \cite{tseng2003comparison} se realiza una comparación de \ac{EBGM} con \ac{PCA} donde el primero obtiene un porcentaje de aciertos es de 96.2\%, el cual es significativamente mas alto que el 71.6\% de un método holístico basado en "Eigenfaces"  probados en una colección de 3282 imágenes de rostros. Otro trabajo comparativo es el de Givens\cite{givens2004features} donde \ac{EBGM} y \ac{PCA} son analizados en 11 factores covariables que intervienen en el proceso de reconocimiento estos factores son raza, genero, edad, el uso de lentes, bello facial, flequillos, el estado de la boca, tez, estado de los ojos, uso de maquillaje y expresión facial. Este trabajos uso dos modelos estadísticos para las comparaciones uno es \ac{ANOVA} y el otro es regresión logística, cuyos resultados indican que sujetos con los ojos cerrados son mas difíciles de reconocer con \ac{EBGM}, esto se debe a que se necesita la posición de los ojos para calcular el resto de los puntos fiduciales, si estos primeros falla indican un punto muy débil del algoritmo, las demás debilidades son relacionadas a este problema ya que se tratan de cambios en los ojos o expresiones muy distorsionadas.

En lo que respecta a vídeo vigilancia existen varios trabajos que abordan el problema de vídeo vigilancia como el "survey" de Gorodnichy sobre las tecnología comerciales para vídeo vigilancia \cite{gorodnichy2014survey} hecha para el gobierno de Canadá para el control de pasos fronterizos, y sus propuestas para aplicar el reconocimiento de rostros en vídeo \cite{gorodnichy2005video}. También existen varias bases de datos para vídeo vigilancia aunque la mayoría se centran en reconocimiento de formas humanas ,descripción de acciones o búsqueda de acciones violentas, existe una base de datos llamada "SCface surveillance cameras face database" \cite{grgic2011scface} que contiene datos de 4,160 imágenes estáticas de 130 sujetos, tomadas a partir de vídeos de vigilancia. También propone un protocolo de prueba basado en \ac{PCA}, la cual para el momento que se realizo esta investigación era una base de datos en venta.

Un tema muy relacionado al reconocimiento de rostros es la detección, cabe diferenciar que reconocimiento y detección a pesar de ser procesos muy ligados en esta tesis solo se ocupa del reconocimiento, pero se menciona algunos trabajos de detección ya que forman parte de la propuesta que se explica en el capitulo 4. A continuación listamos algunos trabajos que están relacionados a la propuesta y a \ac{EBGM}.

\begin{itemize}
\item Face Detection, Pose Estimation, and Landmark Localization in the Wild \cite{zhu2012face}.- Es una propuesta para guardar los puntos de marca necesarios para el reconocimiento en una estructura de tipo árbol con
el fin de lograr reconocimiento en tiempo real.
\item Elastic Bunch Graph Matching Based Face Recognition Under Varying Lighting, Pose, and Expression Conditions \cite{bhat2015elastic}.- En este trabajo se presentan las pruebas de rendimiento de \ac{EBGM} bajo la
variación de luz, pose y expresión todo esto usando la base de pruebas FERRET.
\item Enhance the Alignment Accuracy of Active Shape Models Using Elastic Graph Matching \cite{zhao2004enhance}.- Es una propuesta que usa \ac{ASM} para mejorar la forma de deteccion de punto fiduciales junto con una forma simplificada de \ac{EBGM}.
\item OpenFace: an open source facial behavior analysis toolkit\cite{Baltrusaitis2016}.- Presenta un toolkit que contiene la mejora del ``Cambridge Face Tracker'' en lo que respecta al "Landmark Detector".
\item Robust Facial Landmarking for Registration\cite{salah2007robust}.- Es una comparacion de metodos para encontrar ``Landmarks'' en rostros.
\item Aguará: An Improved Face Recognition Algorithm through Gabor Filter Adaptation\cite{aguerrebere2007aguara}.- Es una implementacion de \ac{EBGM} donde la ecuación que produce los Gabor Wavelet es modificada con análisis de potencial espectral.
\end{itemize}
